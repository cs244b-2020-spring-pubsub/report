In the process of development, the researchers tried different strategies to improve the efficiency. And finally decided to apply the following strategies to Pub/Sub service:

\begin{itemize}
  \item Discard the commit related fields and implementations in Raft. As we previously said, all implementation only handles storage at memory, there will be no data commit in the process.
  \item Create different threads for each heartbeat RPC and request vote RPC. In the first version, all the RPCs in the service are implemented in a blocking way so that the researchers can make sure all the processes come in sequence and are working as expected. However, in performance test, the researchers find that there maybe some nodes in the cluster are slower than the others and thus, the leader will be blocked by the slowest node in the cluster. Then the researchers decided to create a new thread for heartbeat request and vote request, which came out to improve the efficiency a lot. The researchers also tried to create a new thread for append entry request. But during tests, the subscribers may receive messages out of order, which is not acceptable. So the researchers decided to only create new threads for the sidecar transactions.
  \item Start the initial election immediately rather than wait for the timeout. In Raft implementation, the nodes only start a new election if it times out. This will be meaningless when we initialize the cluster. In this implementation, the node will send out a vote request once it finishes the setup and after a random waiting time. Which reduce the initialization time a lot.
  \item Wait a random time before send out a vote request. In the first implementation, the node will send out the vote request once it is timed out. However, during tests, the researchers found that the nodes may be a contention scenario during election process. So in the later version, the node will send out the vote request only after waiting a random time (between 0 and heartbeat timeout).
\end{itemize}