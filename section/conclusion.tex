Scaling remains one of the hardest problems in computer science. Through this project, the researchers learned that different scaling approaches each has its own comparative advantages: master-slave architecture could be one way to extend system performance; meanwhile leader election could be another approach to improve system reliability. However, learning the advantages is not yet enough, the researchers also developed some thinking on how Pub/Sub and scaling could be further optimized in future.

First, the testing strategy adapted throughout the research mostly focused on subscribing side. Subscribe have to keep a long-lasting socket with Pub/Sub servers, but publish is more of a short-lived request but could take place frequently. Given current architectures, where all publish requests are essentially routed to a leader machine, when the publish traffic is at high level the performance of the system could be very different. Hence, how to scale up publish requests still remains an interesting problem that can be better addressed.

Second, after this project the researchers have the idea that the future of scaling solutions should be hybrid of multiple approaches. The essence of a distributed system is the topology it forms; therefore the system users should be able to manipulate such structure however they want to most efficiently solve the problems they need to tackle. Hence, if a distributed system solution could be configured to easily adapt multiple replication strategies, it is likely to be preferred by more users. While developing this system, the researchers attempted to extract the scaling functions as a set of standard interfaces, and the side-car services could just implement the interface. If this goal were to be achieved, the Pub/Sub system the researchers have designed could be able to provide multiple means of replication strategies on a single node. But this idea was eventually unapproachable due to the fact that Pub/Sub implementation needs to be written, such as rerouting publish requests to masters, in different replication strategies so that the strategy could work. In order to achieve the proposed idea, the Pub/Sub base API should be better designed so it has the flexibility to fit the requirements of replication strategies.

Finally, there are also a few features proposed but not yet implemented due to time constraints. The researchers plan to leave them for future development:

\begin{itemize}
    \item \textbf{Authentication checking}: Currently, the implementation only allow one time subscribe, and the user don't need to provide any identification information. In the future, we may consider to add authentication checking in the service. Before subscriber subscribe a topic, the identification information is checked and the publisher may decide to publish the message to specific subset of subscribers that should have the permission to the message.
    \item \textbf{Client revive}: In current version, if a client lost connection with the server, it will miss the message in the meantime and have no way to retrieve those messages. After authentication checking is implemented, we may consider to store the last reply from the subscriber, and retry sending the messages after the subscriber lost connection. If the subscriber login again with the same account, then it should first received all the messages during the time that it is offline.
    \item \textbf{Message history query}: It will be handy if the subscribers can check all the message history based on the timestamp and topic. So that the client will not need to implement a client side log if they do need the history. We may also consider to implement this functionality.
\end{itemize}
